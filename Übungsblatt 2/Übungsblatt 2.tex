
% ----------------------- TODO ---------------------------
% Diese Daten müssen pro Blatt angepasst werden:
\newcommand{\NUMBER}{2}
\newcommand{\EXERCISES}{3}
% Diese Daten müssen einmalig pro Vorlesung angepasst werden:
\newcommand{\COURSE}{Algorithmen}
\newcommand{\TUTOR}{Jan Splett}
\newcommand{\STUDENTA}{Sarah Ertel}
\newcommand{\STUDENTB}{Patrick Greher}
\newcommand{\STUDENTC}{Eugen Ljavin}
\newcommand{\DEADLINE}{03.05.2018}
% ----------------------- TODO ---------------------------

%Template 
\documentclass[a4paper]{scrartcl}
\usepackage[utf8]{inputenc}
\usepackage[ngerman]{babel}
\usepackage{geometry,forloop,fancyhdr,fancybox,lastpage}
\geometry{a4paper,left=3cm, right=3cm, top=3cm, bottom=3cm}

%Math
\usepackage{amsmath,amssymb,tabularx}

%Figures
\usepackage{graphicx,tikz,color,float}
\usetikzlibrary{shapes,trees}

%Algorithms
\usepackage[ruled,linesnumbered]{algorithm2e}

%Kopf- und Fußzeile
\pagestyle {fancy}
\fancyhead[L]{Tutor: \TUTOR}
\fancyhead[C]{\COURSE}
\fancyhead[R]{\today}

\fancyfoot[L]{}
\fancyfoot[C]{}
\fancyfoot[R]{Seite \thepage}

%Formatierung der Überschrift, hier nichts ändern
\def\header#1#2{
  \begin{center}
    {\Large\textbf {\"U}bungsblatt #1}\\
    {(Abgabetermin #2)}
  \end{center}
}

%Definition der Punktetabelle, hier nichts ändern
\newcounter{punktelistectr}
\newcounter{punkte}
\newcommand{\punkteliste}[2]{%
  \setcounter{punkte}{#2}%
  \addtocounter{punkte}{-#1}%
  \stepcounter{punkte}%<-- also punkte = m-n+1 = Anzahl Spalten[1]
  \begin{center}%
  \begin{tabularx}{\linewidth}[]{@{}*{\thepunkte}{>{\centering\arraybackslash} X|}@{}>{\centering\arraybackslash}X}
      \forloop{punktelistectr}{#1}{\value{punktelistectr} < #2 } %
      {%
        \thepunktelistectr &
      }
      #2 &  $\Sigma$ \\
      \hline
      \forloop{punktelistectr}{#1}{\value{punktelistectr} < #2 } %
      {%
        &
      } &\\
      \forloop{punktelistectr}{#1}{\value{punktelistectr} < #2 } %
      {%
        &
      } &\\
    \end{tabularx}
  \end{center}
}

\begin{document}

\begin{tabularx}{\linewidth}{m{0.2 \linewidth}X}
  \begin{minipage}{\linewidth}
    \STUDENTA\\
    \STUDENTB\\
    \STUDENTC
  \end{minipage} & \begin{minipage}{\linewidth}
    \punkteliste{1}{\EXERCISES}
  \end{minipage}\\
\end{tabularx}

\header{Nr. \NUMBER}{\DEADLINE}

% ----------------------- TODO ---------------------------
% Hier werden die Aufgaben/Lösungen eingetragen:

\section*{Aufgabe 1}
\subsection*{a)}
$A=\left[ 0, 2, 4, 6, 8, 10 \right]$ \\
$x=4$

\subsection*{b)}
$A=\left[ 0, 2, 4, 6, 8, 10 \right]$ \\
$x=2$

\subsection*{c)}
\begin{algorithm}[H]
 function ternarySearch$(array\left[ ~ \right], key, lowerBound, upperBound)$ \\
  \If{$upperBound \geq lowerBound$}{
   indexLeft $\gets$ (upperBound - lowerBound) / 3 + lowerBound\;
	 indexRight $\gets$ (upperBound - lowerBound) / 3 + indexLeft\;
   \If{$array\left[indexLeft\right] = key$}{
		return indexLeft\;
		}
		\If{$array\left[indexRight\right] = key$}{
		return indexRight\;
		}
		\If{$array\left[indexLeft\right] > key$}{
		return $ternarySearch(array, key, lowerBound, indexLeft - 1)$\;
		}
		\If{$array\left[indexRight\right] < key$}{
		return $ternarySearch(array, key, indexRight + 1, upperBound)$\;
		}
		return $ternarySearch(array, key, indexLeft + 1, indexRight - 1);$
   }
	 return -1\;
 \caption{Ternary Search}
\end{algorithm}

\begin{enumerate}
	\item 
	Als Eingabeparameter werden das zu durchsuchende Array $(array\left[ ~ \right])$, das gesuchte Element $(key)$ sowie die untere $(lowerBound)$ und obere $(upperBound)$ Grenze verlangt.
	$lowerBound$ sowie $upperBound$ werden als Eingabeparameter benötigt, um einen rekursiven Aufruf zu ermöglichen. Beim erstmaligen (nicht rekursiven) Aufruf, sollte für $lowerBound$ der Wert $0$ und für $upperBound$ der Wert $array.length-1$ übergeben weden. \\ 
	Der Algorithmus liefert den Index des zu suchenden Elements zurück, falls es im Array vorhanden ist. Anderenfalls wird -1 zurückgegeben.
	
	\item
	Abbruchbedingung
	
	\item
	Berechnung der Grenze des ersten und zweiten Drittels.
	
	\item
	Berechnung der Grenze des zweiten und dritten Drittels.
	
	\item
	Überprüfung, ob das gesuchte Element dem Element an Index $indexLeft$ entspricht. 
	
	\item \setcounter{enumi}{7}
	Element wurde gefunden, somit Rückgabe von $indexLeft$
	
	\item
	Überprüfung, ob das gesucht Element dem Element an Index $indexRight$ entspricht. 
	
	\item \setcounter{enumi}{10}
	Element wurde gefunden, somit Rückgabe von $indexRight$
	
	\item
	Überprüfung, ob das gesuchte Element im ersten Drittel liegt
	
	\item \setcounter{enumi}{13}
	Rekursiver Aufruf, sodass im ersten Drittel gesucht wird $\Rightarrow$ Grenzen werden von $lowerBound$ sowie $indexLeft$ (exklusiv) gebildet.
	
	\item
	Überprüfung, ob das gesuchte Element im dritten Drittel liegt
	
	\item \setcounter{enumi}{16}
	Rekursiver Aufruf, sodass im dritten Drittel gesucht wird $\Rightarrow$ Grenzen werden von $indexRight$ (exklusiv) sowie $upperBound$ gebildet.
	
	\item \setcounter{enumi}{18}
	Falls das gesuchte Element nicht im ersten und nicht im dritten Drittel liegt, liegt es im zweiten Drittel.\\
	Rekursiver Aufruf, sodass im dritten Drittel gesucht wird $\Rightarrow$ Grenzen werden von $indexLeft$ (exklusiv) sowie $indexRight$ (exklusiv) gebildet.
	
	\item
	Rückgabe des Wertes -1 wenn das gesuchte Element nicht im Array liegt.
	
\end{enumerate}

\subsection*{d)}
Die Rekursionsvorschrift für den Binary Search, der den Suchbereich auf zwei Teilbereiche aufteilt und \textbf{einen} Vergleich benötigt, um die Entscheidung zu treffen in welchem Bereich das gesuchte Element liegt, lautet $T(n) = T(\frac{n}{2}) + 1$ (im worst case) und hat eine Komplexität von $\mathcal{O}(\log_2 n)$ \\
Der Ternary Search teilt den Suchbereich in \textbf{drei} Teile auf und bnötigt zwei Vergleiche (im worst case), um zu enscheiden in welchem Bereich das gesuchte Element liegt. Folglich hat der Ternary Search die Rekursionsvorschrift $T(n) = T(\frac{n}{3}) + 2$. 
Mit Hilfe des Mastertheorems lässt sich folgende Komplexität ermitteln:

\begin{align*}
T(n) &= T\left(\frac{n}{3}\right) + 2 \Rightarrow T\left(\frac{n}{3}\right) + \mathcal{O}(1) \\
\text{Nach dem Mastertheorem gilt: } &a = 1; ~ b = 3; ~ f(n) = \mathcal{O}(1) \\
\text{Es ist: } &\Theta(n^{\log_3 1}) = \Theta(n^0) = \Theta(1) = f(n) \\
\text{Nach Fall 2 des Mastertheorems gilt: } &T(n) = \mathcal{O}(1) \cdot \log_3 n = \mathcal{O}(\log_3 n)
\end{align*}

Da gilt $\mathcal{O}(\log_3 n) < \mathcal{O}(\log_2 n)$ ist die Laufzeit des Ternary Sarch im Vergleich zur Laufzeit des Binary Search besser.

\subsection*{e)}
\begin{itemize}
	\item 
	Minimale Anzahl an Vergleichen: Ein Vergleich je Rekursionsschritt, wenn davon ausgegangen wird, dass das Element im ersten Drittel liegen müsste $\Rightarrow \log_3 n$
	
	\item
	Maximale Anzahl an Vergleichen: Zwei Vergleiche je Rekursionsschritt, wenn davon ausgegangen wird, dass das Element im zweiten oder im dritten Drittel liegen müsste $\Rightarrow 2 \cdot \log_3 n$
	
	\item
	Durchschnittliche Anzahl an Vergleichen: 
\end{itemize}

\section*{Aufgabe 2}
\subsection*{a)}

\begin{algorithm}[H]
 function findTwoSummands$(array\left[ ~ \right], z)$ \\
	  \For{$i \gets 0; i < array.length; i \gets i + 1$}{
			x $\gets$ array$\left[i\right]$\;
			\For{$j \gets 0; j < array.length; j \gets j + 1$}{
				y $\gets$ array$\left[j\right]$\;
        \If{$x + y = z \wedge x \neq y$}{
					return x, y\;
				}
			}     
    }
		return -1\;
 \caption{Finde ein $x$ und $y$ mit $x \neq y$, sodass $x + y = z$}
\end{algorithm} 

\vspace{1cm}

Für den Algorithmus werden zwei verschachtelte Schleifen benötigt. Jedes Element des Arrays wird dazu mit jedem Element des Arrays verglichen, ausgenommen sich selbst. Es werden somit $n \cdot (n-1) = n^2 - n$ Schleifendurchläufe (im Worst Case) getätigt, was einer Laufzeit von $\mathcal{O}(n^2)$ entspricht. 

\subsection*{b)}
\begin{algorithm}[H]
 function findTwoSummands$(array\left[ ~ \right], z)$ \\
	  \For{$i \gets 0; i < array.length; i \gets i + 1$}{
			x $\gets$ array$\left[i\right]$\;
			y $\gets$ z - x\;
        \If{$binarySearch(array, y) \neq -1$}{
					return x, y\;
				}
			}
		return -1\;
 \caption{Finde ein $x$ und $y$ mit $x \neq y$, sodass $x + y = z$ mit Binary Search}
\end{algorithm} 

\vspace{1cm}

Der angegebende Algorithmus führt $n$ mal die Binäre Suche durch (im Worst Case). Es ist bekannt, dass die Binäre Suche eine Laufzeit von $\mathcal{O}(\log n)$ hat. Es ergibt sich also eine Gesamtlaufzeit von $\mathcal{O}(n \cdot \log n)$

\subsection*{c)}
\begin{algorithm}[H]
 function findTwoSummands$(array\left[ ~ \right], z)$ \\
		hashTable\;
	  \For{$i \gets 0; i < array.length; i \gets i + 1$}{
			
			x $\gets$ array$\left[i\right]$\;
			y $\gets$ z - x\;
        \If{$hashTable[y] \neq empty$}{
					return x, y\;
				}
				hashTable[x] = x\;
			}
		return -1\;
 \caption{Finde ein $x$ und $y$ mit $x \neq y$, sodass $x + y = z$}
\end{algorithm} 

\vspace{1cm}

Vorausgesetzt wird eine Datenstruktur, die einen Einfüge- und Suchzugriff konstanter Laufzeit, also $\mathcal{O}(1)$, ermöglicht. Eine solche Datenstruktur ist beispielsweise eine Hash Table. Für alle $n$ Element des Arrays wird eine Such- und ggfs. Einfügeoperation durchgefürt womit sich eine Gesamtlaufzeit von $n \cdot \mathcal{O}(1) = \mathcal{O}(n)$ ergibt.




\end{document}
%%% Local Variables:
%%% mode: latex
%%% TeX-master: t
%%% End:
